\documentclass[11pt]{article}\usepackage[]{graphicx}\usepackage[]{color}
% maxwidth is the original width if it is less than linewidth
% otherwise use linewidth (to make sure the graphics do not exceed the margin)
\makeatletter
\def\maxwidth{ %
  \ifdim\Gin@nat@width>\linewidth
    \linewidth
  \else
    \Gin@nat@width
  \fi
}
\makeatother

\definecolor{fgcolor}{rgb}{0.345, 0.345, 0.345}
\newcommand{\hlnum}[1]{\textcolor[rgb]{0.686,0.059,0.569}{#1}}%
\newcommand{\hlstr}[1]{\textcolor[rgb]{0.192,0.494,0.8}{#1}}%
\newcommand{\hlcom}[1]{\textcolor[rgb]{0.678,0.584,0.686}{\textit{#1}}}%
\newcommand{\hlopt}[1]{\textcolor[rgb]{0,0,0}{#1}}%
\newcommand{\hlstd}[1]{\textcolor[rgb]{0.345,0.345,0.345}{#1}}%
\newcommand{\hlkwa}[1]{\textcolor[rgb]{0.161,0.373,0.58}{\textbf{#1}}}%
\newcommand{\hlkwb}[1]{\textcolor[rgb]{0.69,0.353,0.396}{#1}}%
\newcommand{\hlkwc}[1]{\textcolor[rgb]{0.333,0.667,0.333}{#1}}%
\newcommand{\hlkwd}[1]{\textcolor[rgb]{0.737,0.353,0.396}{\textbf{#1}}}%
\let\hlipl\hlkwb

\usepackage{framed}
\makeatletter
\newenvironment{kframe}{%
 \def\at@end@of@kframe{}%
 \ifinner\ifhmode%
  \def\at@end@of@kframe{\end{minipage}}%
  \begin{minipage}{\columnwidth}%
 \fi\fi%
 \def\FrameCommand##1{\hskip\@totalleftmargin \hskip-\fboxsep
 \colorbox{shadecolor}{##1}\hskip-\fboxsep
     % There is no \\@totalrightmargin, so:
     \hskip-\linewidth \hskip-\@totalleftmargin \hskip\columnwidth}%
 \MakeFramed {\advance\hsize-\width
   \@totalleftmargin\z@ \linewidth\hsize
   \@setminipage}}%
 {\par\unskip\endMakeFramed%
 \at@end@of@kframe}
\makeatother

\definecolor{shadecolor}{rgb}{.97, .97, .97}
\definecolor{messagecolor}{rgb}{0, 0, 0}
\definecolor{warningcolor}{rgb}{1, 0, 1}
\definecolor{errorcolor}{rgb}{1, 0, 0}
\newenvironment{knitrout}{}{} % an empty environment to be redefined in TeX

\usepackage{alltt}

\usepackage[utf8]{inputenc}
\usepackage{graphicx}
\usepackage{float}
\usepackage{polski}
\usepackage[a4paper, total={6.5in, 8.5in}]{geometry}
\usepackage{amsmath}
\usepackage{amsfonts}
\usepackage[hidelinks]{hyperref}
\usepackage{tabularx}

\urlstyle{rm}
  
\usepackage{setspace}
\newtheorem{stat}{Statement}
\newtheorem{theorem}{Theorem}
\newtheorem{defi}{Definition}
\newtheorem{lem}[stat]{Lemma}
\newtheorem{ex}{Example}[section]
\newtheorem{fact}{Fact}

\frenchspacing
\IfFileExists{upquote.sty}{\usepackage{upquote}}{}
\begin{document}


\title{Bank Marketing data (with social/economic context)}
\author{Maciej Ma戼㸳ecki}
\maketitle
\abstract{W pliku Bank Marketing data.csv znajduj戼㸹 si攼㹡 dane charakteryzuj戼㸹ce klient昼㸳w pewnego banku oraz kampanie marketingowe skierowane do tych klient昼㸳w. Do戼㸳戼㸹czone s戼㸹 ponadto wska㤼㹦niki spo戼㸳eczne i ekonomiczne. Na podstawie tych danych nale戼㹦y zbudowa攼㸶 model prognozuj戼㸹cy szans攼㹡, 戼㹦e klient w wyniku prowadzonej kampanii za戼㸳o戼㹦y lokat攼㹡 terminow戼㸹.}
\tableofcontents

\newpage

\section{Wprowadzenie}
\subsection{Opis problemu}
W ramach kampani marketingowej organizowanej przez pewien bank w latach mi攼㹡dzy majem 2008 rok, a listopadem 2010 roku, by戼㸳y zbierane informacje na temat klient昼㸳w tego banku. 
Na podstawie tych danych planowane jest przewidzenie, czy i jakie rodzaj klient昼㸳w kupi lokat攼㹡 terminow戼㸹 w tym banku.

\subsection{Opis danych}
Nasze dane zawieraj戼㸹 21 column i 4119 wierszy z danymi. Kolumny mo戼㹦emy podzieli攼㸶 na 3 grupy:

\textbf{I: Zmienne zwi戼㸹zane z danymi klienta bankowego:}

\begin{enumerate}
\item Wiek (age): wiek klienta.
\item Praca (job): rodzaj pracy klienta.
\item Stan cywilny (marital): stan cywilny klienta.
\item Edukacja (education): edukacja klienta.
\item Domy㤼㹣lnie (default): Klient wcze㤼㹣niej domy㤼㹣lnie mia戼㸳 kredyt.
\item Mieszkanie (housing): Klient ma kredyt mieszkaniowy.
\item Po戼㹦yczka (loan): Klient ma osobist戼㸹 po戼㹦yczk攼㹡.
\end{enumerate}


\textbf{II: Zmienne zwi戼㸹zane z ostatnim kontaktem bie戼㹦戼㸹cej kampanii marketingowej:}

\begin{enumerate}
\setcounter{enumi}{7}
\item Kontakt (contact): Typ komunikacji kontaktowej (telefonicznej lub kom昼㸳rkowej).
\item Miesi戼㸹c (month): Ostatni kontakt miesi戼㸹ca roku.
\item Dzie昼㸱 tygodnia (day of week): dzie昼㸱 ostatniego kontaktu tygodnia.
\item Czas trwania (duration): czas trwania ostatniego kontaktu w sekundach. Je㤼㹣li czas trwania wynosi 0, nigdy nie skontaktowali㤼㹣my si攼㹡 z klientem, aby za戼㸳o戼㹦y攼㸶 konto lokaty terminowej.
\item Kampania (campaign): liczba kontakt昼㸳w wykonanych podczas tej kampanii i dla tego klienta
\item Liczba dni (pdays): liczba dni, kt昼㸳re up戼㸳yn攼㹡戼㸳y od ostatniego kontaktu klienta z poprzedniej kampanii (warto㤼㹣攼㸶 liczbowa; 999 oznacza, 戼㹦e klient wcze㤼㹣niej si攼㹡 nie skontaktowa戼㸳)
\item Poprzedni (previous): liczba kontakt昼㸳w wykonanych przed t戼㸹 kampani戼㸹 i dla tego klienta (numerycznie)
\item Poutcome: wynik poprzedniej kampanii marketingowej (kategorycznie: 㠼㸴pora戼㹦ka㤼㸴, 㠼㸴nieistniej戼㸹ca㤼㸴, 㠼㸴sukces㤼㸴)
\end{enumerate}

\textbf{III: Atrybuty kontekstu spo戼㸳ecznego i gospodarczego:}

\begin{enumerate}
\setcounter{enumi}{15}
\item Emp.var.rate: wska㤼㹦nik zmienno㤼㹣ci zatrudnienia - wska㤼㹦nik kwartalny 
\item Cons.price.idx: wska㤼㹦nik cen konsumpcyjnych - wska㤼㹦nik miesi攼㹡czny 
\item Cons.conf.idx: wska㤼㹦nik zaufania konsument昼㸳w - wska㤼㹦nik miesi攼㹡czny 
\item Euribor3m: stawka 3-miesi攼㹡czna euribor - wska㤼㹦nik dzienny 
\item Liczba zatrudnionych (nr employed): liczba pracownik昼㸳w - wska㤼㹦nik kwartalny 
\end{enumerate}


\textbf{Zmienna wyj㤼㹣ciowa (po戼㹦戼㸹dany cel):}

\begin{enumerate}
\setcounter{enumi}{20}
\item y - czy klient subskrybowa戼㸳 lokat攼㹡? (dw昼㸳jkowy: 㠼㸴tak㤼㸴, 㠼㸴nie㤼㸴)






















\end{document}
